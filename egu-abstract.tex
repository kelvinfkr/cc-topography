\documentclass{article}

\begin{document}
	The massive expansion of scientific literature on climate change challenges the Intergovenmental Panel on Climate Change (IPCC)'s ability to assess the science according to its objectives.
	Moreover, the number and variety of papers hinders researchers of the science-policy interface from making objective judgements about those IPCC assessments. This Big Literature problem requires innovative solutions to understanding and assessing the science of climate change.
	In this paper, we present a novel application of a machine-reading approach to model the topical content of papers on climate change. This dynamic topic model provides the basis for a \textit{topography} of the literature. We outline the thematic development of the field, identifying emerging topics in climate change literature, such as coral bleaching in the mid-90s or biochar more recently. Topics are used to analyse which areas of research are better covered by IPCC reports, uncovering an under-representation of solution-focused topics such as those on negative emissions, buildings and urban mitigation.

\end{document}